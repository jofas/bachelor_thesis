\documentclass[twoside,11pt]{article}
\usepackage{jmlr2e}
\usepackage{amsmath}
\usepackage[page]{appendix}
\usepackage{xcolor}
\usepackage[marginparsep=30pt]{geometry}
\usepackage{stmaryrd}
\usepackage{algorithm}
\usepackage{algorithmic}
\usepackage{tikz}
\usepackage{tabu}
\usepackage{listings}
\usepackage{fancyref}
\usepackage{relsize}

\usetikzlibrary{%
    arrows,
    arrows.meta,
    decorations,
    backgrounds,
    positioning,
    fit,
    petri,
    shadows,
    datavisualization.formats.functions,
    calc,
    shapes,
    shapes.multipart,
    matrix
}

\DeclareMathOperator*{\argmax}{arg\,max}

\def\ds{\Lbag z_1,\dots,z_n \Rbag}

\begin{document}
% titlepage {{{
\begin{titlepage}
  \begin{flushleft}
	\vspace*{-1cm}
	\includegraphics[scale=0.05]{TH.png}\\
	\vspace*{1cm}
\end{flushleft}
\begin{center}
\begin{LARGE}
\textbf{%
	Approximating the optimal threshold for an abstaining
  classifier based on a reward function with regression
}
\end{LARGE}
~\\
~\\
~\\
\textit{{\LARGE B}ACHELOR {\LARGE T}HESIS}
~\\
~\\
~\\
\begin{Large}
\begin{tabu} to \textwidth {Xr}
Jonas Fassbender
&\href{mailto:jonas@fassbender.dev}{jonas@fassbender.dev}\\
&11117674
\end{tabu}
\end{Large}
~\\
~\\
~\\
\begin{large}
In the course of studies

\textit{{\Large C}OMPUTER {\Large S}CIENCE}
~\\
~\\
~\\
For the degree of

\textit{{\Large B}ACHELOR OF {\Large S}CIENCE}
~\\
~\\
~\\
Technical University of Cologne

Faculty of Computer Science and Engineering
~\\
~\\
~\\
\begin{tabular}{rl}
  First supervisor: &Prof. Dr. Heinrich Klocke\\
                    &Technical University of Cologne\\
  &\\
  Second supervisor: &Prof. Dr. Fotios Giannakopoulos\\
                     &Technical University of Cologne
\end{tabular}
~\\
~\\
~\\
Overath, July 2019
\end{large}
\end{center}
\end{titlepage}
% }}}

%\begin{abstract}%
%\end{abstract}

%\begin{keywords}
%\end{keywords}

% Introduction {{{
\section{Introduction}

An abstaining classifier
\citep[see e.g.][]{vanderlooy_et_al_2009}---also called a
classifier with reject option
\citep[see e.g.][]{fisher_et_al_2016}---is a kind
of confidence predictor.
It can refuse from making a prediction if its confidence in
the prediction is not high enough.
High enough, in this context, means that the confidence is
greater than a certain---hopefully optimal---threshold.
Optimality is dependent on a performance metric set
beforehand.

This thesis introduces a new kind of method for
approximating the optimal threshold based on a reward
function---better known from reinforcement learning than
from the supervised learning setting
\citep[see e.g.][]{sutton_et_al_2018}.
The method treats the reward function as unknown, making it
a very general approach and giving quite the amount of
freedom in designing the reward function.

In supervised learning the concept that is closest to a
reward function is a cost function and many abstract types
of cost in supervised learning are known
\citep[see][]{turney_2000}.

Probably today's most used methods for obtaining the
optimal threshold for reducing the expected cost of an
abstaining classifier are based on the receiver operating
characteristic (ROC) rule
\citep[see][]{tortella_2000,pietraszek_2005,
  vanderlooy_et_al_2009, guan_et_al_2018}.

The method presented in this thesis is more flexible than
the methods based on the ROC rule and can---depending on
the context of the classification problem---produce results
better interpretable than results from a cost setting
(see Chapter~\ref{sec:example}).
Also it is more natural with multi-class classification
problems than the methods based on the ROC rule, all
assuming binary classification problems, wherefore the
classifiers generated by these methods must be transformed
to multi-class classifiers for non-binary problems.

On the other hand the presented method can suffer from its
very general approach and only produces approximations.
This can result in non-optimal and unstable thresholds.

This thesis first presents a motivational example.
In Chapter~\ref{sec:method} the proposed method is
presented.
After that experiments on data sets from the UCI machine
learning repository \citep[see][]{uci} are discussed.
At last further research ideas are listed and a conclusion
is drawn.
% }}}

% Motivational example {{{
\section{Motivational example}
\label{sec:example}

This chapter will point out the usefulness of abstaining
classifiers in real world application domains where
reliability is key.
It will show an example why the reward setting can improve
readability in some domains.
First another example, for which the cost setting---more
commonly used in supervised learning---comes more natural
is given and the differences are discussed.

Abstaining classifiers---compared to typical classifiers,
which classify every prediction, maybe even without a
confidence value in it (then called a bare
prediction)---can be easily integrated into and enhance
processes where they partially replace some of the decision
making, since they can delegate the abstained predictions
back to the underlying process.
The use of abstaining classifiers in domains where
reliability---in regard to prediction errors---is
important, has an interesting aspect in giving
reliability while still being able to decrease work, cost,
etc.\ to some degree.
This is a valuable property if there does not exist a
typical classifier good enough to fully replace the
underlying process.

Many real world application domains for abstaining
classifiers can express a cost function associated to the
decisions about predicting and abstaining of the
classifier---which then chooses the threshold with which it
produces the least amount of cost, therefore minimizing
the cost of introducing the abstaining classifier to the
process.

For example, the real world application domain could be a
facial recognition system at a company which regulates
which employee can enter a trust zone and which can not.
The process which should be enhanced with the facial
recognition system is a manual process where the employee
has to fill out a form in order to receive a key which
opens the trust zone.

In this example, the costs of miss-classifying an
unauthorized person as authorized can be huge for
the company while abstaining or classifying an authorized
employee as unauthorized produces quite low costs---the
authorized employee just has to start the manual process,
which should be replaced by the facial recognition system.

On the other hand, for some real world application domains
a reward function based on which the abstaining classifier
chooses the threshold by maximizing the reward---rather
than minimizing the cost---comes more natural.

Such a domain would be the finance industry,
where we often can associate a certain amount of money an
abstaining classifier can produce or safe by supporting the
decision making of an underlying process.

An example for such a process would be the process of a
bank for granting a consumer credit.
The bank requests information about the consumer from a
credit bureau in order to assess the consumer's credit
default risk.
Now the bank wants to predict the consumer's credit
default risk based on information the bank has about the
consumer.
If the credit default risk is very high or very low the
bank can save money not making a request to the credit
bureau for this consumer.
The optimal threshold for the abstaining classifier making
the prediction about the credit default risk can easily be
expressed by a reward function.
Every correct decision saves the bank the money the request
to the credit bureau costs.
Every miss-classification costs the bank either the amount
of money it would gain by granting the credit, or the
money it loses by giving a credit to somebody that does not
pay the rates.
Abstention cost is the cost of making a request to the
credit bureau.

Using a reward function---like in the example
above---instead of a cost function has an advantage in
readability. One can easily assess the gain of introducing
the abstaining classifier to the process.
Is the reward generated by the abstaining classifier higher
than zero, the process is enhanced by the abstaining
classifier.
Otherwise the abstaining classifier would produce more
cost than it would save and it is not valuable for the
bank to introduce it to its process of assessing a
consumer's credit default risk.
% }}}

% Proposed method based on reward {{{
\section{Proposed method based on reward}
\label{sec:method}

Let $\textbf{X}$ be our observation space and $\textbf{Y}$
our label space. $|\textbf{Y}| < \infty$ since only
classification is discussed. Let $\textbf{Z}$ be the
cartesian product of $\textbf{X}$ and $\textbf{Y}$:
$\textbf{Z} := \textbf{X} \times \textbf{Y}$.
$\textbf{Z}$ is called the example space.
Let an example $z_i$ from $\textbf{Z}$ be:
$z_i := (x_i, y_i); z_i \in \textbf{Z}$.
A data set\footnote{not an actual set but a multi-set since
it can contain the same element more often than one time.}
containing examples $z_1,\dots,z_n$ is annotated as
$\ds$.

A classical machine learning predictor---in the previous
chapter called a typical classifier---can be represented
as a function:
\begin{align}
  \label{eq:D}
  D: \textbf{Z}^* \times \textbf{X} \rightarrow \textbf{Y}.
\end{align}
Its first argument being a data set with an arbitrary
length the classifier is trained on, while the second is an
observation which should be predicted
(mapped to a label from $\textbf{Y}$).

Let $D_{\ds}$ be a classical machine
learning predictor trained on the data set $\ds$ and
let $D_{\ds}(x)$ be equivalent to (\ref{eq:D}).

The proposed method relies on scoring classifiers.
A scoring classifier does not return just
a label but instead returns some score for each label from
our label space.
The only constraint on the scores is that higher scores
are better than lower.
A score could be a probability or just an uncalibrated
confidence value \citep[see][]{vanderlooy_et_al_2009}.

Let $S$ be a scoring classifier:
\begin{align*}
  S: \textbf{Z}^* \times \textbf{X} \rightarrow
     (\textbf{Y} \rightarrow \mathbb{R}).
\end{align*}
$S$ takes the same arguments as (\ref{eq:D}) but instead
of producing bare predictions it returns a function which
maps every label from the label space to a score
determined by $S$.

The method proposed is only interested in the highest
score and the associated label. For that two functions
$k$ and $v$ are defined:
\begin{align*}
  k(S_{\ds}, x) &= \argmax_{y \in \textbf{Y}}
    S_{\ds}(x)(y) \\
  v(S_{\ds}, x) &= \max_{y \in \textbf{Y}}
    S_{\ds}(x)(y).
\end{align*}

The composition $kv$ of $k$ and $v$ returns the tuple with
the label mapped to the highest score:
\begin{align*}
  kv(S_{\ds}, x) = (k(S_{\ds}, x), v(S_{\ds}, x)).
\end{align*}

% relation to reinforcement learning
%The whole idea
% meta architecture
%The idea

%The proposed method relies on an meta architecture
%comparable to the one described in
%\citet{smirnov_et_al_2009}.

%The authors define

% reward function

% algorithm

% }}}

\section{Experiments}

%\section{Other methods for abstaining}
%\label{sec:abs_methods}

\section{Further research}

\section{Conclusion}

\renewcommand{\appendixpagename}{}
\begin{appendices}
  \section*{Appendix}

  \section{Plots}

\end{appendices}

\bibliography{thesis.bib}

\newpage
\section*{Erklärung}
%\markboth{Erklärung}{Erklärung}\addcontentsline{toc}{section}{Erklärung}
Ich versichere, die von mir vorgelegte Arbeit
selbstst\"andig verfasst zu haben.
Alle Stellen, die w\"ortlich oder sinngem\"a{\ss} aus
ver\"offentlichten oder nicht ver\"offentlichten Arbeiten
anderer oder der Verfasserin/des Verfassers selbst
entnommen sind, habe ich als entnommen kenntlich gemacht.
S\"amtliche Quellen und Hilfsmittel, die ich für die Arbeit
benutzt habe, sind angegeben.
Die Arbeit hat mit gleichem Inhalt bzw. in wesentlichen
Teilen noch keiner anderen Pr\"ufungsbeh\"orde vorgelegen.

~\\
~\\
\noindent
\rule{0.35\textwidth}{0.4pt}
\hspace*{3cm}
\rule{0.45\textwidth}{0.4pt}
\newline
Ort, Datum	\hspace*{6.3cm}	Rechtsverbindliche Unterschrift
\end{document}
